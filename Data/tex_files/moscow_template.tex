\documentclass[a5paper, 12pt, twoside]{article}

\input{ {{preambule}} }

\begin{document}
%---------------обложка----------------------------------------------------
    \thispagestyle{empty}
    
    {\centering\bf%
    Федерация спортивного туризма России\\Федерация спортивного туризма – объединение туристов Москвы\par}
    \vspace{1cm}

    {\hfil\includegraphics[height=3.6cm]{ {{pictures['FSTR_logo']}} }\hfil\raisebox{0.1cm}{\includegraphics[height=3.4cm]{ {{pictures['FSTM_logo']}} }}\hfil\par}
    \vspace{1cm}

    {\centering\Large\bf%
    МАРШРУТНАЯ КНИЖКА № 1\par}

    {\centering%
    ТУРИСТСКОГО МАРШРУТА\par}
    \vspace{1.5cm}
        
    {\centering%
    Участники настоящего спортивного мероприятия\\
    находятся под защитой:\par}
    \vspace{0.7cm}

    {\centering%
    Конституции Российской Федерации,\\
    Федерального закона РФ №132-ФЗ от 24 ноября 1996 года <<Об основах туристской деятельности в Российской Федерации>>,\\
    Федерального закона РФ №339-ФЗ от 04 декабря 2007 года <<О физической культуре и спорте в Российской Федерации>>,\\
    Федерации спортивного туризма России\par}
    \vspace{3cm}

    {\centering%
    {{general_info['year']}} г.\par}

\newpage %---------------первая страница-----------------------------------

    {\centering\large\bf%
    ВНИМАНИЕ!\par}

    Согласно действующему законодательству, туристские группы должны проинформировать
    службы МЧС России за 10 дней до начала путешествия. При подаче онлайн-заявки на
    регистрацию необходимо указать состав и количество участников, руководителей,
    контактные телефоны, подробную информацию о маршруте и т.д. Ответственный сотрудник
    ведомства в субъекте РФ обязан обработать заявку в течение одного рабочего дня,
    после чего передать сведения в спасательное подразделение, в зоне ответственности
    которого планируется маршрут. Для этого необходимо:

    \begin{enumerate}[leftmargin=0.5cm, rightmargin=0cm]
        \item Пройти онлайн регистрацию туристской группы на сайте\hfil\\
        {\small\url{https://forms.mchs.ru/service/registration_tourist_groups}},\\
        отделение МЧС по региону прохождения маршрута.
        \item Получить регистрационный номер группы в МЧС и сообщить о
        номере регистрации в региональном отделении МЧС в МКК.
        \item Сообщить в территориальный орган МЧС и МКК, а также в случае
        участия в официальных соревнованиях в ГСК, о выходе на маршрут.
        \item По окончании маршрута проинформировать территориальный орган
        МЧС в срок и способом, указанном в уведомлении.
    \end{enumerate}

    \setlength\parindent{0pt}

\newpage %---------------вторая страница-----------------------------------

    \section{Общие сведения}\label{s:group}
        Группа туристов \textbf{%
        {{general_info['club_name']}}, {{general_info['city']}}}
        в составе \textbf{%
        {{general_info['size_of_group']}}}
        человек
        с \textbf{%
        {{general_info['start_day']}}}
        по \textbf{%
        {{general_info['finish_day']}}} совершает
        прохождение \textbf{горного} маршрута \textbf{%
        {{general_info['complexity']}}} категории сложности
        в районе: \textbf{%
        {{general_info['mountanious_area']}}}
        \vspace{0.5cm}
        
        по маршруту: \textit{%
            {{general_info['rout']}}
        }
        \vspace{2.5cm}
        
        \hbox to \textwidth{Руководитель группы:\hfil\textbf{%
        {{general_info['leader_full_name']}}}}

        \hbox to \textwidth{\hfil тел.:
        {{general_info['leader_phone_number']}}}
        \vspace{1.5cm}
        
        Телефон для связи
        
        \hbox to \textwidth{с группой на маршруте:\hfil
        {{general_info['group_phone_number']}}}
        \vspace{0.5cm}
        
        \hbox to \textwidth{Телефон МЧС:\hfil
        {{general_info['mchs_phone_number']}}}

        \hbox to \textwidth{\hfil\parbox{0.6\textwidth}{\raggedleft%
        {{general_info['mchs_name']}}\\{{general_info['mchs_addres']}}\par}}

\newpage %---------------третья страница-----------------------------------

    \section{Состав группы}
        \newcommand{\myfootnotetext}{\tiny Даю своё согласие на обработку, использование и хранение персональных
        данных, согласно Федеральному закону 152-ФЗ от 27.07.2006 <<О персональных данных>>,
        необходимых для рассмотрения маршрутных и отчетных документов.}

        \renewcommand{\arraystretch}{1.2}
        \setlength{\arrayrulewidth}{0.25mm}
        \setlength{\arraycolsep}{1pt}
        \setlength{\tabcolsep}{0pt}
        {\scriptsize%
		\begin{longtable}{%
            |>{\centering\arraybackslash} m{0.5cm}%
            |>{\centering\arraybackslash} m{3cm}%
            |>{\centering\arraybackslash} m{1.5cm}%
            |>{\centering\arraybackslash} m{6cm}%
            |>{\centering\arraybackslash} m{1.5cm}|}
			\hline\rowcolor{Gray}
			№   &   ФИО							    &   Год рожд.       &   Место проживания: субъект РФ, населённый пункт, телефон &   {Подпись\footnote{\myfootnotetext}} \\ \hline
			{{tables['group_members_1']}}
		\end{longtable}}

        \stepcounter{footnote}
        \footnotetext{\tiny При очном рассмотрении замена участников заверяется штампом МКК, а при
        заочном "--- прикладывается письмо от МКК, направившей маршрутные документы на рассмотрение}

\newpage %---------------четвёртая страница--------------------------------

        \renewcommand{\myfootnotetext}{\tiny По требованию МКК предъявляются справки о пройденных
        маршрутах или список ниток маршрутов, пройденных участниками и руководителем, заверенные
        нижестоящей МКК или другие материалы, подтверждающие туристский опыт.)}
        
        \newcommand{\myfootnotetextt}{\tiny О знании об опасностях для жизни и здоровья при прохождении
        запланированного маршрута, о правах и обязанностях участника туристской группы, а также для
        спортсменов: Кодекса путешественника, Правил вида спорта <<спортивный туризм>>, часть 2.}

        \renewcommand{\arraystretch}{2}
        {\scriptsize%
        \begin{longtable}{%
            |>{\centering\arraybackslash} m{2.9cm}%
            |>{\centering\arraybackslash} m{5.3cm}%
            |>{\centering\arraybackslash} m{3.2cm}%
            |>{\centering\arraybackslash} m{1.5cm}|}
            \hline
            \rowcolor{Gray}
            Телефон, E-mail, контактные данные родственников    &   Туристский опыт\footnote{\myfootnotetext} {\tiny перечислить маршруты, совершенные по данному виду туризма с указанием районов и категорий сложности}   &%
            Обязанности в группе, распределение по средствам сплава &   Подпись\footnote{\myfootnotetextt}  \\ \hline
            {{tables['group_members_2']}}
        \end{longtable}}

        {\centering\textbf{<<Наличие туристского опыта подтверждаю>>}\par}
        \vspace{0.3cm}

        Член МКК \quad\hbox to 0.25\textwidth{\leaders\hrule height 0.2pt \hfill}%
        \rlap{\raisebox{-0.3cm}{\hbox to 0.45\textwidth{\hfil\scriptsize Фамилия И.О.\hfil}}}%
        \hbox to 0.45\textwidth{\;(\;\leaders\hrule height 0.2pt \hfill\;)}

\newpage %---------------пятая страница------------------------------------
    \stepcounter{section}
    \subsection[]{График движения группы по маршруту\\ (заявленный)}\label{ss:general_plan}
        
        \renewcommand{\arraystretch}{1.2}
        {\scriptsize%
        \begin{longtable}{%
            |>{\centering\arraybackslash} m{1cm}%
            |>{\centering\arraybackslash} m{0.8cm}%
            |>{\centering\arraybackslash} m{7cm}%
            |>{\centering\arraybackslash} m{1.3cm}%
            |>{\centering\arraybackslash} m{2.1cm}|}
            \hline\rowcolor{Gray}
            Дата    &   День пути   &   Участки маршрута                                                                &    Км      &   Способы передвижения\\ \hline
            {{tables['general_rout']}}
        \end{longtable}}

        {\small%
        \textbf{Итого} активными способами передвижения: \textbf{ {{general_info['total_distance']}}\,км}}

\newpage %---------------шестая страница-----------------------------------
    \newgeometry{top = 0.2 cm, bottom = 1 cm, left = 1.4 cm, right = 1.4 cm}
    \thispagestyle{empty}
    \subsection[]{Изменения графика движения по маршруту (согласованные с МКК)\protect\footnote{\tiny При внесении изменений, в п. \ref{ss:general_plan} записывают те дни, в которых произведены изменения. Если маршрут согласован без изменений, то делают запись <<Без изменений>>.}}
        \vspace{-0.6cm}

        {\scriptsize%
        \begin{longtable}{%
            |>{\centering\arraybackslash} m{1cm}%
            |>{\centering\arraybackslash} m{0.8cm}%
            |>{\centering\arraybackslash} m{7cm}%
            |>{\centering\arraybackslash} m{1.3cm}%
            |>{\centering\arraybackslash} m{2.1cm}|}
            \hline\rowcolor{Gray}
            Дата            &   День пути   &   Участки маршрута                                                                                                                                &   Км      &   Способы передвижения\\ \hline
                            &               &                                                                                                                                                   &           &                       \\ \hline
                            &               &                                                                                                                                                   &           &                       \\ \hline
                            &               &                                                                                                                                                   &           &                       \\ \hline
                            &               &                                                                                                                                                   &           &                       \\ \hline
                            &               &                                                                                                                                                   &           &                       \\ \hline
        \end{longtable}}
        \vspace{-1cm}
     
    \subsection[]{График движения группы по запасному варианту}
        \vspace{-0.6cm}

        {\scriptsize%
        \begin{longtable}{%
            |>{\centering\arraybackslash} m{1cm}%
            |>{\centering\arraybackslash} m{0.8cm}%
            |>{\centering\arraybackslash} m{7cm}%
            |>{\centering\arraybackslash} m{1.3cm}%
            |>{\centering\arraybackslash} m{2.1cm}|}
            \hline\rowcolor{Gray}
            Дата            &   День пути   &   Участки маршрута                                                                                                                                &   Км      &   Способы передвижения\\ \hline
            {{tables['reserve_rout']}}
        \end{longtable}}
        \vspace{-1cm}

    \subsection[]{Аварийные выходы с маршрута}
        \vspace{-0.6cm}

        {\scriptsize%
        \begin{longtable}{%
            |>{\centering\arraybackslash} m{1cm}%
            |>{\centering\arraybackslash} m{0.8cm}%
            |>{\centering\arraybackslash} m{7cm}%
            |>{\centering\arraybackslash} m{1.3cm}%
            |>{\centering\arraybackslash} m{2.1cm}|}
            \hline\rowcolor{Gray}
            Дата            &   День пути   &   Участки маршрута                                                                                                                                &   Км      &   Способы передвижения\\ \hline
            {{tables['emergency_rout']}}
        \end{longtable}}
        {\small%
        \textbf{Итого} активными способами передвижения: \textbf{ {{general_info['total_reserve_distance']}}\,км}}
   
\newpage %---------------седьмая страница----------------------------------
    \restoregeometry
    \section[]{Схема маршрута\protect\footnote{На схеме, желательно в цветном исполнении, наносится маршрут движения (основной, запасной, аварийный), даты и места предполагаемых мест ночлегов. Представленная схема должна давать четкое представление о нитке прохождения маршрута, его определяющих препятствий. По требованию МКК, к маршрутной книжке прилагается картографический материал, предполагаемый для использования группой на маршруте.}}
        
        Прилагается.

\newpage %---------------восьмая страница----------------------------------
    
    \section[]{Сложные участки маршрута и способы их преодоления}
        
        {\small%
        \begin{enumerate}
            {{general_info['difficult_parts']}}
        \end{enumerate}}

\newpage %---------------девятая страница----------------------------------
    \section{Материальное обеспечение группы}\label{s:weight}
        \footnotesize
        Необходимый набор продуктов питания \uline{имеется}.

        Групповое и личное снаряжение в достаточном количестве \uline{имеется}.

        {\centering\textbf{Специальное снаряжение}\par}
        \vspace{-0.3cm}

        \renewcommand{\arraystretch}{1.1}
        {\footnotesize%
        \begin{longtable}{%
            |>{\centering\arraybackslash} m{4.8cm}%
            |>{\centering\arraybackslash} m{1.3cm}%
            |>{\centering\arraybackslash} m{4.8cm}%
            |>{\centering\arraybackslash} m{1.3cm}|}
            \hline\rowcolor{Gray}
            \multicolumn{2}{|c|}{Групповое}                     &   \multicolumn{2}{c|}{Личное}             \\ \hline\rowcolor{Gray}
            Наименование                            &   Кол-во  &   Наименование                &   Кол-во  \\ \hline
            {{tables['equipment']}}
        \end{longtable}}
        \vspace{-0.3cm}

        Необходимый ремонтный набор \uline{имеется}.

        Необходимый набор лекарств и материалов в аптечке первой помощи \uline{имеется}.
    
        Картосхема маршрута, перечень определяющих препятствий и способы их прохождения, а также варианты аварийных выходов прилагаются.
        
        {\centering\textbf{Весовые характеристики груза, взятого на маршрут}\par}
        \vspace{-0.3cm}

        {\footnotesize%
        \begin{longtable}{%
            |>{\centering\arraybackslash} m{5.9cm}%
            |>{\centering\arraybackslash} m{3cm}%
            |>{\centering\arraybackslash} m{3.3cm}|}
            \hline\rowcolor{Gray}
            Наименование                            &   На 1 человека       &   На группу в {{general_info['size_of_group']}} чел. \\ \hline
            Продукты (всего/в день)                 &   {{weights[0][0]}}   &   {{weights[0][1]}}   \\ \hline
            Групповое снаряжение                    &   {{weights[1][0]}}   &   {{weights[1][1]}}   \\ \hline
            Личное снаряжение                       &   {{weights[2][0]}}   &   {{weights[2][1]}}   \\ \hline
            \textbf{Всего (за вычетом заброски)}    &   {{weights[3][0]}}   &   {{weights[3][1]}}   \\ \hline
        \end{longtable}}
        \vspace{-0.3cm}

        \settowidth{\tmp}{Максимальная нагрузка на одного мужчину \textbf{ {{weights[4][1]}}\,кг},}
        
        \vbox{%
        \hbox{Максимальная нагрузка на одного мужчину \textbf{ {{weights[4][1]}}\,кг},}
        \hbox to \tmp{\hfil женщину \textbf{ {{weights[4][0]}}\,кг}.}}
        \vspace{0.1cm}
        
        {\scriptsize Сведения, изложенные в разделах 1--6, подтверждаю. Обязуемся соблюдать необходимые меры безопасности при
        прохождении запланированного маршрута, руководствоваться требованиями правил вида спорта <<спортивный туризм>>
        (Часть 2) и Регламента организации и прохождения спортивных туристских маршрутов.}
        
        \textbf{Руководитель похода\;\hbox to 3cm{\leaders\hrule height 0.2pt \hfill}\;({{general_info['leader_short_name']}})}
        \vspace{0.1cm}
        
        \hbox to 0.8\textwidth{\hfil Дата: \textbf{ {{general_info['date']}} г}.}
        \normalsize

\newpage %---------------десятая страница----------------------------------

    \section{Ходатайство МКК}

        \textbf{Председателю МКК}\;%
        \rlap{\raisebox{-0.2cm}{\hbox to 0.62\textwidth{\hfil\tiny Наименование вышестоящей МКК\hfil}}}%
        \hbox to 0.62\textwidth{\leaders\hrule height 0.2pt \hfill}
        \vspace{0.2cm}
        
        \hbox to \textwidth{\strut\leaders\hrule height 0.2pt \hfill}
        \vspace{0.5cm}
        
        В связи с отсутствием полномочий у маршрутно-ква\-ли\-фи\-ка\-ци\-он\-ной комиссии\;%
        \rlap{\raisebox{-0.2cm}{\hbox to 0.7\textwidth{\hfil\tiny Наименование ходатайствующей МКК\hfil}}}%
        \hbox to 0.7\textwidth{\leaders\hrule height 0.2pt \hfill}
        \vspace{0.2cm}
        
        просим Вас рассмотреть маршрутные документы и дать по ним своё заключение. Предварительное
        рассмотрение произведено нашей комиссией.

        \settowidth{\tmp}{}
        
        \textbf{Председатель МКК}\;\hbox to 0.25\textwidth{\leaders\hrule height 0.2pt \hfill}%
        \rlap{\raisebox{-0.3cm}{\hbox to 0.42\textwidth{\hfil\footnotesize Фамилия И.О.\hfil}}}%
        \hbox to 0.42\textwidth{\;(\;\leaders\hrule height 0.2pt \hfill\;)}
        \vspace{0.5cm}
        
        \hfill<<\vbox{\hbox{\hphantom{999}}\hrule height 0.2pt}>>\;\hbox to 3cm{\leaders\hrule height 0.2pt\hfil}\;%
        20\,\vbox{\hbox{\hphantom{999}}\hrule height 0.2pt}\,г.\hspace{-0.2cm}
        \vspace{1.5cm}
        
        Штамп МКК
        \vspace{2cm}
        
        Адрес МКК:\;\hrulefill
        \vspace{0.5cm}
        
        Тел./факс:\;\hrulefill
        \vspace{0.5cm}

        e-mail:\;\hrulefill\;(обязателен)
        \vspace{0.5cm}
        
        ФИО председателя МКК\;\hrulefill\,

\newpage %---------------одиннадцатая страница-----------------------------

    \section[]{Результаты рассмотрения в маршрутно-\\квалификационной комиссии}

        Маршрутно-квалификационная комиссия \textbf{ФСТ-ОТМ}
        
        в составе\;\leaders\hrule height 0.2pt\hfill\strut
        \vspace{0.2cm}
        
        \,\leaders\hrule height 0.2pt\hfill\strut
        \vspace{0.2cm}
        
        с участием\;\leaders\hrule height 0.2pt\hfill\strut
        \vspace{0.1cm}
        
        рассмотрев материалы заявленного похода группы под руководством {{general_info['leader_short_name_pp']}} считает, что (ненужное зачеркнуть):

        \begin{enumerate}
            \item Маршрут \textbf{соответствует} (не соответствует) заявленной категории сложности.
            \item Туристский опыт руководителя группы \textbf{соответствует} (не соответствует) технической сложности маршрута.
            \item Туристский опыт участников группы \textbf{соответствует} (не соответствует) технической сложности маршрута.
            \item Заявочные материалы \textbf{отвечают} (не отвечают) установленным требованиям.
            \item Другие замечания:\;\leaders\hrule height 0.2pt\hfill\strut
            
            \,\leaders\hrule height 0.2pt\hfill\strut

            \,\leaders\hrule height 0.2pt\hfill\strut

            \,\leaders\hrule height 0.2pt\hfill\strut
        \end{enumerate}
        
        Группе назначается (\textbf{не назначается}) контрольная проверка на местности\;\leaders\hrule height 0.2pt\hfill\strut
        \vspace{0.2cm}
        
        \,\leaders\hrule height 0.2pt\hfill\strut
        \vspace{0.2cm}
        
        \,\leaders\hrule height 0.2pt\hfill\strut
        \vspace{0.2cm}
        
        \,\leaders\hrule height 0.2pt\hfill\strut
        
        \hbox to \textwidth{\hfil\raisebox{1pt}{\scriptsize(где, когда и по каким вопросам)}\hfil}

\newpage %---------------двенадцатая страница------------------------------

    \section[]{Результаты проверки группы на\\ местности}

        \settowidth{\tmp}{Группа в составе: руководитель}
        \ttmp = \textwidth
        \addtolength{\ttmp}{-\tmp}
        
        Группа в составе: руководитель%
        \rlap{\raisebox{-0.3cm}{\hbox to \ttmp{\hfil\scriptsize Фамилия И.О.\hfil}}}%
        \;\leaders\hrule height 0.2pt\hfill\strut
        \vspace{0.2cm}

        \settowidth{\tmp}{участники}
        \ttmp = \textwidth
        \addtolength{\ttmp}{-\tmp}
        
        участники%
        \rlap{\raisebox{-0.3cm}{\hbox to \ttmp{\hfil\scriptsize Фамилия И.О.\hfil}}}%
        \;\leaders\hrule height 0.2pt\hfill\strut
        \vspace{0.2cm}
        
        \,\leaders\hrule height 0.2pt\hfill\strut
        \vspace{0.2cm}
        
        прошла проверку\;%
        <<\vbox{\hbox{\hphantom{999}}\hrule height 0.2pt}>>\;\hbox to 2.5cm{\leaders\hrule height 0.2pt\hfil}\;%
        20\,\vbox{\hbox{\hphantom{999}}\hrule height 0.2pt}\,г.%
        \rlap{\raisebox{-0.3cm}{\hbox to 0.28\textwidth{\hfil\scriptsize место проведения\hfil}}}%
        \hbox to 0.28\textwidth{\;\leaders\hrule height 0.2pt \hfill}
        \vspace{0.05cm}
        
        по следующим вопросам:\;\leaders\hrule height 0.2pt\hfill\strut
        \vspace{0.2cm}
        
        \,\leaders\hrule height 0.2pt\hfill\strut
        \vspace{0.2cm}
        
        Результаты проверки:\;\leaders\hrule height 0.2pt\hfill\strut
        \vspace{0.2cm}
        
        \,\leaders\hrule height 0.2pt\hfill\strut
        \vspace{0.2cm}
        
        Проверяющий\;\hbox to 0.3\textwidth{\leaders\hrule height 0.2pt \hfill}%
        \rlap{\raisebox{-0.3cm}{\hbox to 0.42\textwidth{\hfil\scriptsize Фамилия И.О.\hfil}}}%
        \hbox to 0.42\textwidth{\;(\;\leaders\hrule height 0.2pt \hfill\;)}
        \vspace{0.1cm}

    \section[]{Заключение маршрутно-\\квалификационной комиссии}

        Группа под руководством \textbf{%
        {{general_info['leader_full_name_pp']}}} имеет (не имеет) положительное
        заключение МКК о прохождении планируемого маршрута.
        \vspace{0.2cm}
        
        Особые указания:\;\leaders\hrule height 0.2pt\hfill\strut
        \vspace{0.2cm}
        
        \,\leaders\hrule height 0.2pt\hfill\strut
        \vspace{0.2cm}
        
        \,\leaders\hrule height 0.2pt\hfill\strut
        \vspace{0.2cm}
        
        \,\leaders\hrule height 0.2pt\hfill\strut
        \vspace{0.2cm}
        
        Срок сдачи отчета о походе до\;%
        <<\vbox{\hbox{\hphantom{999}}\hrule height 0.2pt}>>\;\hbox to 3cm{\leaders\hrule height 0.2pt\hfil}\;%
        20\,\vbox{\hbox{\hphantom{999}}\hrule height 0.2pt}\,г.%
        
        в объеме \textbf{письменный, полный}
        \vspace{0.2cm}
        
        Адреса и реквизиты для связи с региональным органом МЧС, ПСС или ПСО в районе проведения маршрута:%
        \;\leaders\hrule height 0.2pt\hfill\strut
        \vspace{0.2cm}

        \,\leaders\hrule height 0.2pt\hfill\strut
        \vspace{0.2cm}

\newpage %---------------тринадцатая страница------------------------------

    \section{Контрольные пункты и сроки}

        О прохождении маршрута группа должна сообщить:

        \begin{enumerate}[itemsep=1pt]
            \item \parbox[t]{0.58\textwidth}{%
            {{general_info['mchs_name']}}} \hfil \parbox[t]{0.325\textwidth}{\raggedleft{{general_info['mchs_phone_number']}}}
            {{general_info['report_to']}}
        \end{enumerate}

        \begin{itemize}[label=из, noitemsep]
            \item {{general_info['start_report_point']}} до {{general_info['start_report_date']}}\;г.
            \item {{general_info['finish_report_point']}} до {{general_info['finish_report_date']}}\;г.
        \end{itemize}

        Телефон для связи с группой на маршруте:\hfill{{general_info['group_phone_number']}}

        Время и график сеансов связи:\;\hrulefill

        Номер телефона координатора группы:\;\hrulefill

        Электронная почта координатора группы:\;\hrulefill
        \vspace{0.4cm}

        \hbox to \textwidth{%
        \textbf{Председатель МКК}\hfil\hbox to 0.2\textwidth{\leaders\hrule height 0.2pt \hfill}%
        \rlap{\raisebox{-0.3cm}{\hbox to 0.42\textwidth{\hfil\scriptsize Фамилия И.О.\hfil}}}%
        \hbox to 0.42\textwidth{\;(\;\leaders\hrule height 0.2pt \hfill\;)}}
        \vspace{0.4cm}

        \hbox to \textwidth{%
        \textbf{Члены МКК}\hfil\hbox to 0.2\textwidth{\leaders\hrule height 0.2pt \hfill}%
        \rlap{\raisebox{-0.3cm}{\hbox to 0.42\textwidth{\hfil\scriptsize Фамилия И.О.\hfil}}}%
        \hbox to 0.42\textwidth{\;(\;\leaders\hrule height 0.2pt \hfill\;)}}
        \vspace{0.2cm}

        \hbox to \textwidth{\hfil%
        \hbox to 0.2\textwidth{\leaders\hrule height 0.2pt \hfill}%
        \rlap{\raisebox{-0.3cm}{\hbox to 0.42\textwidth{\hfil\scriptsize Фамилия И.О.\hfil}}}%
        \hbox to 0.42\textwidth{\;(\;\leaders\hrule height 0.2pt \hfill\;)}}
        \vspace{0.2cm}

        \hbox to \textwidth{\hfil%
        \hbox to 0.2\textwidth{\leaders\hrule height 0.2pt \hfill}%
        \rlap{\raisebox{-0.3cm}{\hbox to 0.42\textwidth{\hfil\scriptsize Фамилия И.О.\hfil}}}%
        \hbox to 0.42\textwidth{\;(\;\leaders\hrule height 0.2pt \hfill\;)}}
        \vspace{0.2cm}

        {\small%
        \textbf{Судья по виду:} <<Сертификаты на знание антидопинговых правил и страховые полисы предъявлены,
        меддопуск имеется. Группа допущена к соревнованиям>>%
        \rlap{\raisebox{-0.3cm}{\hbox to 0.65\textwidth{\hfil\scriptsize Статус и наименование соревнований\hfil}}}%
        \hbox to 0.65\textwidth{\;\leaders\hrule height 0.2pt \hfill}}
        \vspace{0.5cm}

        Штамп МКК
        \vspace{0.2cm}

        \hfill%
        \rlap{\raisebox{-0.3cm}{\hbox to 0.3\textwidth{\hfil\scriptsize подпись\hfil}}}%
        \hbox to 0.3\textwidth{\leaders\hrule height 0.2pt \hfill}%
        \rlap{\raisebox{-0.3cm}{\hbox to 0.3\textwidth{\hfil\scriptsize Фамилия И.О.\hfil}}}%
        \hbox to 0.3\textwidth{\quad\leaders\hrule height 0.2pt \hfill}
        \vspace{0.5cm}

        \hfill<<\vbox{\hbox{\hphantom{999}}\hrule height 0.2pt}>>\;\hbox to 3cm{\leaders\hrule height 0.2pt\hfil}\;%
        20\,\vbox{\hbox{\hphantom{999}}\hrule height 0.2pt}\,г.%

\newpage %---------------пятнадцатая страница------------------------------

    \section[]{Регистрация в территориальном органе\\ МЧC}

        \textbf{Регистрационный номер}\;\leaders\hrule height 0.2pt\hfill\strut
        \vspace{0.4cm}

        \textbf{Дополнительные отметки}\;\leaders\hrule height 0.2pt\hfill\strut
        \vspace{0.2cm}

        \hbox to \textwidth{%
        \;\leaders\hrule height 0.2pt \hfill}
        \vspace{0.2cm}

        \hbox to \textwidth{%
        \;\leaders\hrule height 0.2pt \hfill}
        \vspace{1cm}

        \settowidth{\tmp}{Штамп ПСС (ПСО)}

        \hfill\hbox{\vbox{\hbox{Штамп ПСС (ПСО)}
        \rlap{\raisebox{0cm}{\hbox to \tmp{\hfil\footnotesize (при наличии)\hfil}}}}}
        \vspace{2cm}

    \section{Решение МКК о зачёте маршрута}

        Пройденный группой \textbf{горный} маршрут под руководством {{general_info['leader_short_name_pp']}} оценен
        \;\leaders\hrule height 0.2pt \hfill\;категорией сложности.
        \vspace{1cm}

        \hfill Справки выданы в количестве\;\hbox to 2cm{\leaders\hrule height 0.2pt \hfill}\;шт.
        \vspace{2cm}

        \hbox to \textwidth{%
        \textbf{Председатель МКК}\hfil\rlap{\raisebox{-0.3cm}{\hbox to 0.3\textwidth{\hfil\scriptsize подпись\hfil}}}%
        \hbox to 0.3\textwidth{\leaders\hrule height 0.2pt \hfill}%
        \rlap{\raisebox{-0.3cm}{\hbox to 0.3\textwidth{\hfil\scriptsize Фамилия И.О.\hfil}}}%
        \hbox to 0.3\textwidth{\;(\;\leaders\hrule height 0.2pt \hfill\;)}}
        \vspace{1cm}

        \hbox to 0.35\textwidth{\hfil Штамп МКК}
        \vspace{1cm}

        \hbox to \textwidth{\hfil<<\vbox{\hbox{\hphantom{999}}\hrule height 0.2pt}>>\;\hbox to 3cm{\leaders\hrule height 0.2pt\hfil}\;%
        20\,\vbox{\hbox{\hphantom{999}}\hrule height 0.2pt}\,г.}

\newpage %---------------шестнадцатая страница-----------------------------

    \strut

\newpage %---------------семнадцатая страница------------------------------

    {\centering%
    \textbf{Положение о МКК\\
    (Извлечение)}\par}
    \vspace{1cm}

    \textbf{5.7. При рассмотрении заявочных документов на походы МКК обязаны
    проверить:}

    \begin{itemize}[itemsep=2pt, leftmargin=1.1cm, label=$\triangleright$]
        \item разработку маршрута и график движения группы по основному и запасным вариантам,
        наличие картографических материалов;
        \item знание руководителем группы района похода условий передвижения и естественных препятствий в нем;
        \item соответствие туристского опыта руководителя и участников похода заявленному маршруту;
        \item правильность подбора группой снаряжения, продовольствия, медикаментов;
        \item намеченные группой меры по обеспечению безопасности при проведении похода;
        \item правильность выбора контрольных пунктов и сроков;
    \end{itemize}

    Заявочные документы регистрируются и хранятся в организации, при которой создана МКК, не менее трех лет.
    \vspace{0.5cm}

    \textbf{5.8. МКК имеют право:}

    \begin{itemize}[itemsep=2pt, leftmargin=1.1cm, label=$\triangleright$]
        \item вызвать участников группы и проверить знание ими Правил проведения туристских
        спортивных походов, вопросов техники и тактики похода;
        \item назначить группе контрольный выход, где проверяется умение пользоваться снаряжением,
        преодолевать естест\-венные препятствия и действовать в аварийных ситуациях.
    \end{itemize}

\newpage %---------------восемнадцатая страница---------------------------

\strut

\newpage %---------------девятнадцатая страница---------------------------

% Пустая

\newpage %---------------двадцатая страница-------------------------------

\strut

\newpage %---------------двадцать первая страница--------------------------

\strut

\end{document}