\documentclass[a5paper, 12pt, twoside]{article}

%%%----------СЛУЖЕБНОЕ----------%%%

\usepackage{iftex}
\usepackage{cmap}
\usepackage{fixme}

%%%----------ШРИФТЫ----------%%%
\usepackage{amsmath, amsthm, wasysym}

\ifxetex
	\usepackage[charter]{mathdesign}
	\usepackage{polyglossia}
	\usepackage{fontenc}
    \usepackage{fontspec}

	\setmainlanguage{russian}
	\setotherlanguage{english}
	
	% \newfontfamily\russianfont[Script=Cyrillic]{Raleway}
	\setmainfont{ITC Charter Pro}
\else
	\usepackage[charter]{mathdesign}
	\usepackage[T2A]{fontenc}
	\usepackage[utf8]{inputenc}
	\usepackage[russian, english]{babel}
	
	\renewcommand*{\rmdefault}{cmr}
	
\fi

%%%----------ГИПЕРССЫЛКИ----------%%%

\usepackage{hyperref}

\hypersetup{
	linktoc=all,
	colorlinks=true,
	linkcolor=Blue,
	urlcolor=Mulberry,
	citecolor=OliveGreen,
	filecolor=Mulberry
}

%%%----------ПОЛЯ----------%%%

\usepackage[top = 1 cm, bottom = 1 cm, left = 1.4 cm, right = 1.4 cm]{geometry}
\usepackage{indentfirst}

%%%----------КОЛОНТИТЫЛЫ----------%%%

\usepackage{fancyhdr}
\renewcommand{\headrule}{}
\setlength{\headheight}{1cm}
\pagestyle{fancy}
\fancyhead[OR]{Маршрутная книжка №\;\hbox to 2cm{\leaders \hrule height 0.2pt\hfill}}
\fancyhead[E, OC, OL]{}

%%%----------ГРАФИКА----------%%%

\usepackage{graphicx}
\usepackage[section]{placeins}
\usepackage[dvipsnames, table]{xcolor}

\definecolor{Gray}{gray}{0.9}

%%%----------ТАБЛИЦЫ----------%%%

\usepackage{array}
\usepackage{longtable}

\setlength{\arrayrulewidth}{0.25mm}
\setlength{\arraycolsep}{1pt}
\setlength{\tabcolsep}{0pt}

%%%----------СПИСКИ----------%%%

\usepackage[inline]{enumitem}

%%%----------МОИ ДЛИНЫ----------%%%
\usepackage{setspace}

\newlength{\tmp}
\newlength{\ttmp}


\usepackage{ulem}


\begin{document}
%---------------обложка----------------------------------------------------
    \thispagestyle{empty}
    
    {\centering\bf%
    Федерация спортивного туризма России\\Федерация спортивного туризма – объединение туристов Москвы\par}
    \vspace{1cm}

    {\hfil\includegraphics[height=3.6cm]{ FSTR_logo.png }\hfil\raisebox{0.1cm}{\includegraphics[height=3.4cm]{ FSTM_logo.png }}\hfil\par}
    \vspace{1cm}

    {\centering\Large\bf%
    МАРШРУТНАЯ КНИЖКА №\phantom{10000}\par}

    {\centering%
    ТУРИСТСКОГО МАРШРУТА\par}
    \vspace{1.5cm}
        
    {\centering%
    Участники настоящего спортивного мероприятия\\
    находятся под защитой:\par}
    \vspace{0.7cm}

    {\centering%
    Конституции Российской Федерации,\\
    Федерального закона РФ №132-ФЗ от 24 ноября 1996 года <<Об основах туристской деятельности в Российской Федерации>>,\\
    Федерального закона РФ №339-ФЗ от 04 декабря 2007 года <<О физической культуре и спорте в Российской Федерации>>,\\
    Федерации спортивного туризма России\par}
    \vspace{3cm}

    {\centering%
    2025 г.\par}

\newpage %---------------первая страница-----------------------------------
    \strut
%     {\centering\large\bf%
%     ВНИМАНИЕ!\par}

%     Согласно действующему законодательству, туристские группы должны проинформировать
%     службы МЧС России за 10 дней до начала путешествия. При подаче онлайн-заявки на
%     регистрацию необходимо указать состав и количество участников, руководителей,
%     контактные телефоны, подробную информацию о маршруте и т.д. Ответственный сотрудник
%     ведомства в субъекте РФ обязан обработать заявку в течение одного рабочего дня,
%     после чего передать сведения в спасательное подразделение, в зоне ответственности
%     которого планируется маршрут. Для этого необходимо:

%     \begin{enumerate}[leftmargin=0.5cm, rightmargin=0cm]
%         \item Пройти онлайн регистрацию туристской группы на сайте\hfil\\
%         {\small\url{https://forms.mchs.ru/service/registration_tourist_groups}},\\
%         отделение МЧС по региону прохождения маршрута.
%         \item Получить регистрационный номер группы в МЧС и сообщить о
%         номере регистрации в региональном отделении МЧС в МКК.
%         \item Сообщить в территориальный орган МЧС и МКК, а также в случае
%         участия в официальных соревнованиях в ГСК, о выходе на маршрут.
%         \item По окончании маршрута проинформировать территориальный орган
%         МЧС в срок и способом, указанном в уведомлении.
%     \end{enumerate}

    \setlength\parindent{0pt}

\newpage %---------------вторая страница-----------------------------------

    \section{Общие сведения}\label{s:group}
        Группа туристов \textbf{%
        ТК МГУ, г. Москва}
        в составе \textbf{%
        9}
        человек
        с \textbf{%
        25.7.2025}
        по \textbf{%
        16.8.2025} совершает
        прохождение \textbf{горного} маршрута \textbf{%
        четвертой} категории сложности
        в районе: \textbf{%
        Памир, Рушанский хребет}
        \vspace{0.5cm}
        
        по маршруту: \textit{%
            пос. Бардара} --- д.р. Бардара --- д.р. Мана --- пер. Ах-ба (4796, 1Б) --- пер. Выхыныч ниж. (4573, н/к) --- д.р. Бартан- ги --- д.р. Уар --- лед. Уар --- пер. Куйбышевских авиаторов (5200, 2А) --- пер. п/п (4970, 2А) --- лед. Сафедоб --- вер. 5480 (п/в, 1Б*) --- пер. Сафедоб Верхн. (5250, 2А*)--- лед. Ю.Девлох --- пер. Е. Иванова (5300) + рад. вер. Кадат (5467) + пер. Ген (5140, в связке 2Б) --- лед. Ген --- пер. Альпинистов (5150) + рад. вер. 5470 (п/в, в связке 2Б) --- лед. Удариф --- пер. Раумид (5250, 2Б) --- лед. Штеклозар --- д.р. Патхур --- пос. Патхур --- д.р. Окмамод- дара --- пер. Рог Тройной (5370, 2Б) --- пер. п/п (5300, 2А) --- лед. Варшез --- рад. п. Скалистый (5700, 2Б) --- пер. Айсберг (4980) + п/п (5040, в связке 1Б) --- д.р. Кулендара --- д.р. Дузахдара
        
        \vspace{1.5cm}
        
        \hbox to \textwidth{Руководитель группы:\hfil\textbf{%
        Чашникова Анастасия Александровна}}

        \hbox to \textwidth{\hfil тел.:
        79508163944}
        \vspace{1cm}
        
        \parbox[t]{0.4\textwidth}{\raggedright Телефон координатора группы:} \hfill \parbox[t]{0.5\textwidth}{\raggedleft Гриша (до 9.08) +79032748883 \\ @byabik,\\\vspace{0.2cm} Катя (после 9.08) +79258523626 \\ @Kropocheva\_Ekaterina}
        \vspace{0.5cm}
        
        \hbox to \textwidth{Телефон МЧС:\hfil
        (83522) 2-40-57, 2-53-47}

        \hbox to \textwidth{\hfil\parbox{0.6\textwidth}{\raggedleft%
        Штаб КЧС по ГБАО\\город Хорог, улица Саидамир Абдурахмонова, 28\par}}

\newpage %---------------третья страница-----------------------------------

    \section{Состав группы}
        \newcommand{\myfootnotetext}{\tiny Даю своё согласие на обработку, использование и хранение персональных
        данных, согласно Федеральному закону 152-ФЗ от 27.07.2006 <<О персональных данных>>,
        необходимых для рассмотрения маршрутных и отчетных документов.}

        \renewcommand{\arraystretch}{1.2}
        \setlength{\arrayrulewidth}{0.25mm}
        \setlength{\arraycolsep}{1pt}
        \setlength{\tabcolsep}{0pt}
        {\scriptsize%
		\begin{longtable}{%
            |>{\centering\arraybackslash} m{0.5cm}%
            |>{\centering\arraybackslash} m{3cm}%
            |>{\centering\arraybackslash} m{1.5cm}%
            |>{\centering\arraybackslash} m{6cm}%
            |>{\centering\arraybackslash} m{1.5cm}|}
			\hline\rowcolor{Gray}
			№ & ФИО							        & Год рожд. & Место проживания: субъект РФ, населённый пункт, телефон   &   {Подпись\footnote{\myfootnotetext}} \\ \hline
			1 & Чашникова Анастасия Александровна   & 1998      & г. Москва, 79508163944                                    &\rule{0cm}{1.2cm} \\ \hline
            2 & Цимбалов Юрий Александрович         & 1989      & г. Москва, 79689461405                                    &\rule{0cm}{1.2cm} \\ \hline
            3 & Ткачёв Алкесей Владимирович         & 1988      & г. Москва, 79266119794                                    &\rule{0cm}{1.2cm} \\ \hline
            4 & Большаков Иван Дмитриевич           & 1996      & г. Москва, 79261174887                                    &\rule{0cm}{1.2cm} \\ \hline
            5 & Зимина Софья Николаевна             & 1989      & г. Москва, 79099643782                                    &\rule{0cm}{1.2cm} \\ \hline
            6 & Михайлов Дмитрий Сергеевич          & 1995      & г. Москва, 79661817755                                    &\rule{0cm}{1.2cm} \\ \hline
            7 & Короткова Дарья Алексеевна          & 2002      & г. Москва, 79775467530                                    &\rule{0cm}{1.2cm} \\ \hline
            8 & Коротков Дмитрий Юрьевич            & 2001      & г. Москва, 79773805536                                    &\rule{0cm}{1.2cm} \\ \hline
            9 & Кутов Данил Константинович          & 1990      & г. Москва, 79269055996                                    &\rule{0cm}{1.2cm} \\ \hline 
		\end{longtable}}

        \stepcounter{footnote}
        \footnotetext{\tiny При очном рассмотрении замена участников заверяется штампом МКК, а при
        заочном --- прикладывается письмо от МКК, направившей маршрутные документы на рассмотрение}

\newpage %---------------четвёртая страница--------------------------------

        \renewcommand{\myfootnotetext}{\tiny По требованию МКК предъявляются справки о пройденных
        маршрутах или список ниток маршрутов, пройденных участниками и руководителем, заверенные
        нижестоящей МКК или другие материалы, подтверждающие туристский опыт.)}
        
        \newcommand{\myfootnotetextt}{\tiny О знании об опасностях для жизни и здоровья при прохождении
        запланированного маршрута, о правах и обязанностях участника туристской группы, а также для
        спортсменов: Кодекса путешественника, Правил вида спорта <<спортивный туризм>>, часть 2.}

        \renewcommand{\arraystretch}{2}
        {\scriptsize%
        \begin{longtable}{%
            |>{\centering\arraybackslash} m{0.5cm}%
            |>{\centering\arraybackslash} m{2.9cm}%
            |>{\centering\arraybackslash} m{4.9cm}%
            |>{\centering\arraybackslash} m{3.2cm}%
            |>{\centering\arraybackslash} m{1.5cm}|}
            \hline
            \rowcolor{Gray}
            № & Телефон, E-mail, контактные данные родственников        & Туристский опыт\footnote{\myfootnotetext} {\tiny перечислить маршруты, совершенные по данному виду туризма с указанием районов и категорий сложности} & Обязанности в группе, распределение по средствам сплава &   Подпись\footnote{\myfootnotetextt}  \\ \hline
            1 & 79775206991,\newline Алла Николаевна,\newline свекровь  & 4 с эл. 5 ГУ (Терскей, Кокшаал-Тоо, Фанские горы), 3ГР (Центральный Кавказ) & руководитель            &\rule{0cm}{1.2cm} \\ \hline
            2 & 79775206991,\newline Алла Николаевна,\newline мать      & 4ГУ (Заалайский хребет, Фанские горы)                                       & завпит                  &\rule{0cm}{1.2cm} \\ \hline
            3 &                                                         & 3ГУ (Центральный Кавказ)                                                    & штурман                 &\rule{0cm}{1.2cm} \\ \hline
            4 & 79166867285,\newline Оксана Леонидовна,\newline мать    & 4ГУ (Высокий Алай)                                                          & снаряженец              &\rule{0cm}{1.2cm} \\ \hline
            5 & 79652347030,\newline Алексей Евгеньевич,\newline муж    & 4ГУ (Кокшаал-Тоо)                                                           & медик                   &\rule{0cm}{1.2cm} \\ \hline
            6 & 79637176767,\newline Сергей Дмитриевич,\newline отец    & 4ГУ (Кокшаал-Тоо, Фанские горы, Акшийрак)                                   & финансист, культуролог  &\rule{0cm}{1.2cm} \\ \hline
            7 & 79015904187,\newline Елена Геннадьевна,\newline мать    & 3ГУ (Центральный Кавказ)                                                    & завпит, худ. фотограф   &\rule{0cm}{1.2cm} \\ \hline
            8 & 79036186560,\newline Юрий Николаевич,\newline отец      & 3ГУ (Зап. Тянь-Шань, Центральный Кавказ), 1ГР (Центральный Кавказ)          & реммастер               &\rule{0cm}{1.2cm} \\ \hline
            9 & 79266528170,\newline Елена Николаевна,\newline жена     & 4ГУ (Кокшаал-Тоо)                                                           &                         &\rule{0cm}{1.2cm} \\ \hline 
        \end{longtable}}

        {\centering\textbf{<<Наличие туристского опыта подтверждаю>>}\par}
        \vspace{0.3cm}

        Член МКК \quad\hbox to 0.25\textwidth{\leaders\hrule height 0.2pt \hfill}%
        \rlap{\raisebox{-0.3cm}{\hbox to 0.45\textwidth{\hfil\scriptsize Фамилия И.О.\hfil}}}%
        \hbox to 0.45\textwidth{\;(\;\leaders\hrule height 0.2pt \hfill\;)}

\newpage %---------------пятая страница------------------------------------
    \stepcounter{section}
    \subsection[]{График движения группы по маршруту\\ (заявленный)}\label{ss:general_plan}
        
        \renewcommand{\arraystretch}{1.2}
        {\scriptsize%
        \begin{longtable}{%
            |>{\centering\arraybackslash} m{1cm}%
            |>{\centering\arraybackslash} m{0.8cm}%
            |>{\centering\arraybackslash} m{6cm}%
            |>{\centering\arraybackslash} m{1.1cm}%
            |>{\centering\arraybackslash} m{2cm}%
            |>{\centering\arraybackslash} m{2.1cm}|}
            \hline\rowcolor{Gray}
            Дата & День пути    & Участки маршрута                                                                  & Км                &   Набор / Сброс   & Способы передвижения\\ \hline
            24.7 & 0            & переезд из Душанбе в Бардару                                                      &                   &                   & Транспорт\\ \hline
            25.7 & 1            & подъем по д.р. Бардара                                                            & 4                 &   +200 / -0       & Пешком\\ \hline
            26.7 & 2            & подъем по д.р. Бардара                                                            & 12.5              &   +600 / -0       & Пешком\\ \hline
            27.7 & 3            & подъем по д.р. Бардара, занос заброски                                            & 3,5 + 7,3 рад.    &   +570 / -470     & Пешком\\ \hline
            28.7 & 4            & подъем по д.р. Мана, ледовые занятия                                              & 3+1,5 рад.        &   +550 / -500     & Пешком\\ \hline
            29.7 & 5            & пер. Ахба (4796, 1Б) -- пер. Выхынч ниж. (4573, н/к) -- Верховья Бартанги         & 15                &   +900 / -350     & Пешком\\ \hline
            30.7 & 6            & д.р. Бартанги                                                                     & 7                 &   +0 / -500       & Пешком\\ \hline
            31.7 & 7            & д.р. Уар                                                                          & 10                &   +800 / -0       & Пешком\\ \hline
            1.8  & 8            & лед. Уар -- пер. Куйбышевских авиаторов (5200, 2А)                                & 10                &   +600 / -300     & Пешком\\ \hline
            2.8  & 9            & забор заброски -- пер. п/п (4970, 2А)                                             & 0,5+4рад.         &   +0 / -300       & Пешком\\ \hline
            3.8  & 10           & долина притока Сафедоба -- лед. Сафедоб                                           & 10                &   +530 / -930     & Пешком\\ \hline
            4.8  & 11           & лед. Сафедоб -- вер. 5470 (п/в, рад., 1Б) -- Сафедоб Верхн. (5250, 2А*)           & 5,3 +1,2 рад.     &   +800 / -200     & Пешком\\ \hline
            5.8  & 12           & пер. Е. Иванова (5300) + рад. вер. Кадат (5467) + пер. Ген (5140, в связке 2Б)    & 3.8               &   +300 / -1100    & Пешком\\ \hline
            6.8  & 13           & пер. Ген (5140, в связке 2Б) -- В склон пер. Альпинистов                          & 8.3               &   +730 / -300     & Пешком\\ \hline
            7.8  & 14           & пер. Альпинистов (5150, 2Б) -- рад. п. п/п (5470, н/к)                            & 1+0,7 рад         &   +720 / -300     & Пешком\\ \hline
            8.8  & 15           & З склон пер. Альпинистов -- д.р. Равмеддара -- лед. Удариф                        & 10                &   +800 / -1350    & Пешком\\ \hline
            9.8  & 16           & пер. Раумид (5250, 2Б) -- лед. Штеклозар                                          & 2                 &   +630 / -130     & Пешком\\ \hline
            10.8 & 17           & лед. Штеклозар -- д.р. Патхур                                                     & 10                &   +0 / -1500      & Пешком\\ \hline
            11.8 & 18           & д.р. Патхур -- пос. Патхур                                                        & 10                &   +0 / -600       & Пешком\\ \hline
            12.8 & 19           & пос. Патхур -- д.р. Окмамоддара -- С склон пер. Рог Тройной                       & 9                 &   +1400 / -0      & Пешком\\ \hline
            13.8 & 20           & С склон пер. Рог Тройной -- С гребень пер. Рог Тройной                            & 3                 &   +750 / -0       & Пешком\\ \hline
            14.8 & 21           & пер. Рог Тройной (5370, 2Б) -- пер. п/п (5300, 2А) -- лед. Варшез                 & 3                 &   +300 / -500     & Пешком\\ \hline
            15.8 & 22           & рад. п. Скалистый (5700, 2Б) -- лед Варшез -- оз. Варшезкуль                      & 2,5+3 рад         &   +750 / -950     & Пешком\\ \hline
            16.8 & 23           & оз. Варшезкуль -- пер. Айсберг (5040, 1Б) -- д.р. Кулендара -- д.р. Дузахдара     & 17                &   +400 / -2200    & Пешком\\ \hline
            17.8 & 24           & переезд в Душанбе                                                                 &                   &                   & Транспорт\\ \hline 
        \end{longtable}}

        {\small%
        \textbf{Итого} активными способами передвижения: \textbf{ 179\,км}}

\newpage %---------------шестая страница-----------------------------------
    \newgeometry{top = 0.2 cm, bottom = 1 cm, left = 1.4 cm, right = 1.4 cm}
    \thispagestyle{empty}
    \subsection[]{Изменения графика движения по маршруту (согласованные с МКК)\protect\footnote{\tiny При внесении изменений, в п. \ref{ss:general_plan} записывают те дни, в которых произведены изменения. Если маршрут согласован без изменений, то делают запись <<Без изменений>>.}}
        \vspace{-0.6cm}

        {\scriptsize%
        \begin{longtable}{%
            |>{\centering\arraybackslash} m{1cm}%
            |>{\centering\arraybackslash} m{0.8cm}%
            |>{\centering\arraybackslash} m{7cm}%
            |>{\centering\arraybackslash} m{1.3cm}%
            |>{\centering\arraybackslash} m{2.1cm}|}
            \hline\rowcolor{Gray}
            Дата            &   День пути   &   Участки маршрута                                                                                                                                &   Км      &   Способы передвижения\\ \hline
                            &               &                                                                                                                                                   &           &                       \\ \hline
        \end{longtable}}
        \vspace{-1cm}
     
    \subsection[]{График движения группы по запасному варианту}
        \vspace{-0.6cm}

        {\scriptsize%
        \begin{longtable}{%
            |>{\centering\arraybackslash} m{1cm}%
            |>{\centering\arraybackslash} m{0.8cm}%
            |>{\centering\arraybackslash} m{7cm}%
            |>{\centering\arraybackslash} m{1.3cm}%
            |>{\centering\arraybackslash} m{2.1cm}|}
            \hline\rowcolor{Gray}
            Дата            &   День пути   &   Участки маршрута                                                                                                                                &   Км      &   Способы передвижения\\ \hline
            1.8 & 8 & лед. Уар - пер. Куйбышевцев (5250, 2А) -- в случае камнеопасности пер. Куйбышевских авиаторов (5200, 2А) & 13 & пешком\\ \hline 2.8 & 9 & пер. 50 лет газеты "КТ" (4960, 2А) & 0,5 + 4 рад. & пешком\\ \hline 4.8 & 11 & пер. Надежда (5000, 2А) -- при отставании от план-графика & 6.6 & пешком\\ \hline 5.8 & 12 & лед. Сафедоб -- пер. Светлый (5380) + вер. Сафед Хайкал (5450, в связке 2Б) -- при радикальном отставании от план-графика & 5 & пешком\\ \hline 6.8 & 13 & лед. Чапдара - д.р. Чапдара & 7 & пешком\\ \hline 7.8 & 14 & д.р. Чапдара - д.р. Патхур & 15 & пешком\\ \hline 9.8 & 16 & лед. Удариф -- пер. 40 лет МИФИ (5020, 2А) -- в случае камнеопасности пер. Раумид, отставании от план-графика & 7.5 & Пешком\\ \hline 10.8 & 17 & пер. Туристов МИФИ (4814, 1Б) -- д.р. Патхур ИЛИ пер. Удариф Ю. (5000, 2А) & 6 & Пешком\\ \hline 12.8 & 19 & пос. Патхур -- д.р. Окмамоддара вплоть до морены -- при отставании от план-графика & 11 & Пешком\\ \hline 13.8 & 20 & пер. п/п (5270, 2А) -- л. Варшез & 3.5 & Пешком\\ \hline 
        \end{longtable}}
        \vspace{-1cm}

    \subsection[]{Аварийные выходы с маршрута}
        \vspace{-0.6cm}

        {\scriptsize%
        \begin{longtable}{%
            |>{\centering\arraybackslash} m{1.3cm}%
            |>{\centering\arraybackslash} m{1cm}%
            |>{\centering\arraybackslash} m{6.5cm}%
            |>{\centering\arraybackslash} m{1.3cm}%
            |>{\centering\arraybackslash} m{2.1cm}|}
            \hline\rowcolor{Gray}
            Дата            &   День пути   &   Участки маршрута                                                                                                                                &   Км      &   Способы передвижения\\ \hline
            25.7--29.7 & 1 -- 5 & Спуск по д.р. Бардара и д.р. Мана & до 27 & пешком\\ \hline 29.7--30.7 & 6 -- 7 & Спуск по д.р. Бартанги -- д.р. Андаравадж & до 35 & пешком\\ \hline 31.7--8.1 & 7 -- 8 & Спуск по д.р. Уар -- д.р. Андаравадж & до 33 & пешком\\ \hline 8.1--8.2 & 8 -- 9 & Спуск по д.р. Бардара & до 33 & пешком\\ \hline 8.3--8.4 & 10 -- 11 & пер. Дружба (4800, 2А) - д.р. Шазуддара (до 20 км) & до 32 & пешком\\ \hline 8.4--8.5 & 11 -- 12 & Спуск по лед. Ю Девлох -- д.р. Девлох & до 26 & пешком\\ \hline 8.5--8.7 & 12 -- 14 & Спуск по лед. Ген -- зап. приток р. Девлох -- д.р. Девлох & до 23 & пешком\\ \hline 8.7--8.9 & 14 -- 16 & Спуск по д.р. Равмеддара & до 23 & пешком\\ \hline 8.9--8.11 & 16 -- 18 & спуск по д.р. Зувордара & до 13  & пешком\\ \hline 8.11--13.8 & 18 -- 20 & спуск по д.р. Окмамоддара & до 13 & пешком\\ \hline 14.8--16.8 & 21 -- 23 & спуск по д.р. Варшездара & до 12 & пешком\\ \hline 
        \end{longtable}}
        {\small%
        \textbf{Итого} активными способами передвижения: \textbf{ 179\,км}}
   
\newpage %---------------седьмая страница----------------------------------
    \restoregeometry
    \section[]{Схема маршрута\protect\footnote{На схеме, желательно в цветном исполнении, наносится маршрут движения (основной, запасной, аварийный), даты и места предполагаемых мест ночлегов. Представленная схема должна давать четкое представление о нитке прохождения маршрута, его определяющих препятствий. По требованию МКК, к маршрутной книжке прилагается картографический материал, предполагаемый для использования группой на маршруте.}}
        
        Прилагается.

\newpage %---------------восьмая страница----------------------------------
    
    \section[]{Сложные участки маршрута и способы их преодоления}
        
        {\footnotesize%
        \begin{enumerate}
            \item Подъем на п/п восточне пер. 50 лет газеты КТ. Сн-лед склон до 40 градусов, наверху обход большой поперечной трещины. Общий набор ок. 200м. Связки, при необходимости од\-но\-вре\-мен\-ное движение с промежуточными точками, или до 3 веревок перил.
            \item Пер. Сафедоб Верхний. Подъем крутизной 30 градусов на 100м, могут быть трудности в отсутствие снега. Связки, возможно одновременное движение с промежуточными точками. На спуск когда-то был небольшой (20м) ледовый кулуар, далее склон до 45 градусов со сбросом 100м, 1-2 больших бергшрунда. Дюльфер на 3-4 веревки.
            \item Спуск с пер. Ген. Крутой ледник от 30 до 60 градусов, местами сложной формы. Общий сброс ок. 250м. Попеременная и перильная страховка. Предположительно, 4-6 дюльферов на полную веревку.
            \item Подъем на пер. Альпинистов. Крутой разорванный ледник до 50 градусов, затем разрушенный ледник, несколько отвесных трещин проходится в лоб. Набор в крутой части взлета до 250м. Связки, перильная страховка (до 8 веревок)
            \item Подъем на пер. Раумид. Крутой ледник до 50 градусов. Набор на обеих ступенях взлета ок. 350м. Возможно камнеопасно по краям, надо внимательно выбирать линию движения. Связки, одновременное движение с промежуточными точками, перильная страховка (до 6 веревок)
            \item Подъем на пер. Рог Тройной. Крутой закрытый ледник до 35 градусов, бергшрунд, затем крутой ледово-снежный гребень до 40 градусов. Связки, одновременное движение с промежуточными точками, попеременная страховка
            \item П/п в гребне между ледниками Рог и Варшез. На подъем сн-лед склон крутизной до 50 градусов, возможны трещины. Связки, попеременная страховка или перильная страховка (количество веревок может значительно колебаться). Спуск по осыпи с крутизной до 50 градусов - возможны заглаженные скальные кулуары, движение плотной группой.
            \item Пик Скалистый. Снежный склон до 35 градусов, ледовый склон до 40 градусов с кальгаспорами, пологий снежный гребень, рыхлый ближе к вершине. Связки, одновременное движение с промежуточными точками (возможна перильная страховка).
        \end{enumerate}}

\newpage %---------------девятая страница----------------------------------
    \section{Материальное обеспечение группы}\label{s:weight}
        \footnotesize
        Необходимый набор продуктов питания \uline{имеется}.

        Групповое и личное снаряжение в достаточном количестве \uline{имеется}.

        {\centering\textbf{Специальное снаряжение}\par}
        \vspace{-0.3cm}

        \renewcommand{\arraystretch}{1.1}
        {\footnotesize%
        \begin{longtable}{%
            |>{\centering\arraybackslash} m{4.8cm}%
            |>{\centering\arraybackslash} m{1.3cm}%
            |>{\centering\arraybackslash} m{4.8cm}%
            |>{\centering\arraybackslash} m{1.3cm}|}
            \hline\rowcolor{Gray}
            \multicolumn{2}{|c|}{Групповое}                     &   \multicolumn{2}{c|}{Личное}             \\ \hline\rowcolor{Gray}
            Наименование                            &   Кол-во  &   Наименование                &   Кол-во  \\ \hline
            Верёвка статич. 50 м, \diameter 9 мм & 3 & Система с блокировкой & 1\\ \hline Веревка динам. 50 м, \diameter 9 мм & 1 & Карабины & 6\\ \hline Расходные концы, \diameter 6 мм & 20 м & Спусковое устройство & 1\\ \hline Ледобуры с карабином & 18 & Прусик короткий & 1\\ \hline Станционные петли с мастеркарабином & 5 & Прусик длинный / корделет & 1\\ \hline петли 60 см & 4 & петля 60 см & 1\\ \hline Лопата & 1 & Каска & 1\\ \hline Рации & 4 & Жумар & 1\\ \hline GPS-навигатор & 2 & Кошки, пара & 1\\ \hline пауэрбанк, 20000 mah & 1 & Ледоруб & 1\\ \hline солнечная батарея & 1 & Трекинговые палки, пара & 1\\ \hline спутниковый трекер & 1 &  & \\ \hline спутниковый тел. турайя & 1 &  & \\ \hline 
        \end{longtable}}
        \vspace{-0.3cm}

        Необходимый ремонтный набор \uline{имеется}.

        Необходимый набор лекарств и материалов в аптечке первой помощи \uline{имеется}.
    
        Картосхема маршрута, перечень определяющих препятствий и способы их прохождения, а также варианты аварийных выходов прилагаются.
        
        {\centering\textbf{Весовые характеристики груза, взятого на маршрут}\par}
        \vspace{-0.3cm}

        {\footnotesize%
        \begin{longtable}{%
            |>{\centering\arraybackslash} m{5.9cm}%
            |>{\centering\arraybackslash} m{3cm}%
            |>{\centering\arraybackslash} m{3.3cm}|}
            \hline\rowcolor{Gray}
            Наименование                            &   На 1 человека       &   На группу в 9 чел. \\ \hline
            Продукты (всего/в день)                 &   13.8\,кг / 0.6\,кг   &   124.2\,кг / 5.4\,кг   \\ \hline
            Групповое снаряжение                    &   6.0\,кг   &   54\,кг   \\ \hline
            Личное снаряжение                       &   13.0\,кг   &   117\,кг   \\ \hline
            \textbf{Всего (за вычетом заброски)}    &   32.8\,кг (25.0\,кг)   &   295.2\,кг (225.0\,кг)   \\ \hline
        \end{longtable}}
        \vspace{-0.3cm}

        \settowidth{\tmp}{Максимальная нагрузка на одного мужчину \textbf{ 26.0\,кг},}
        
        \vbox{%
        \hbox{Максимальная нагрузка на одного мужчину \textbf{ 26.0\,кг},}
        \hbox to \tmp{\hfil женщину \textbf{ 23.0\,кг}.}}
        \vspace{0.1cm}
        
        {\scriptsize Сведения, изложенные в разделах 1--6, подтверждаю. Обязуемся соблюдать необходимые меры безопасности при
        прохождении запланированного маршрута, руководствоваться требованиями правил вида спорта <<спортивный туризм>>
        (Часть 2) и Регламента организации и прохождения спортивных туристских маршрутов.}
        
        \textbf{Руководитель похода\;\hbox to 3cm{\leaders\hrule height 0.2pt \hfill}\;(Чашникова А. А.)}
        \vspace{0.1cm}
        
        \hbox to 0.8\textwidth{\hfil Дата: \textbf{ 13.6.2025 г}.}
        \normalsize

\newpage %---------------десятая страница----------------------------------

    \section{Ходатайство МКК}

        \textbf{Председателю МКК}\;%
        \rlap{\raisebox{-0.2cm}{\hbox to 0.62\textwidth{\hfil\tiny Наименование вышестоящей МКК\hfil}}}%
        \hbox to 0.62\textwidth{\leaders\hrule height 0.2pt \hfill}
        \vspace{0.2cm}
        
        \hbox to \textwidth{\strut\leaders\hrule height 0.2pt \hfill}
        \vspace{0.5cm}
        
        В связи с отсутствием полномочий у маршрутно-ква\-ли\-фи\-ка\-ци\-он\-ной комиссии\;%
        \rlap{\raisebox{-0.2cm}{\hbox to 0.7\textwidth{\hfil\tiny Наименование ходатайствующей МКК\hfil}}}%
        \hbox to 0.7\textwidth{\leaders\hrule height 0.2pt \hfill}
        \vspace{0.2cm}
        
        просим Вас рассмотреть маршрутные документы и дать по ним своё заключение. Предварительное
        рассмотрение произведено нашей комиссией.

        \settowidth{\tmp}{}
        
        \textbf{Председатель МКК}\;\hbox to 0.25\textwidth{\leaders\hrule height 0.2pt \hfill}%
        \rlap{\raisebox{-0.3cm}{\hbox to 0.42\textwidth{\hfil\footnotesize Фамилия И.О.\hfil}}}%
        \hbox to 0.42\textwidth{\;(\;\leaders\hrule height 0.2pt \hfill\;)}
        \vspace{0.5cm}
        
        \hfill<<\vbox{\hbox{\hphantom{999}}\hrule height 0.2pt}>>\;\hbox to 3cm{\leaders\hrule height 0.2pt\hfil}\;%
        20\,\vbox{\hbox{\hphantom{999}}\hrule height 0.2pt}\,г.\hspace{-0.2cm}
        \vspace{1.5cm}
        
        Штамп МКК
        \vspace{2cm}
        
        Адрес МКК:\;\hrulefill
        \vspace{0.5cm}
        
        Тел./факс:\;\hrulefill
        \vspace{0.5cm}

        e-mail:\;\hrulefill\;(обязателен)
        \vspace{0.5cm}
        
        ФИО председателя МКК\;\hrulefill\,

\newpage %---------------одиннадцатая страница-----------------------------

    \section[]{Результаты рассмотрения в маршрутно-\\квалификационной комиссии}

        Маршрутно-квалификационная комиссия \textbf{ФСТ-ОТМ}
        
        в составе\;\leaders\hrule height 0.2pt\hfill\strut
        \vspace{0.2cm}
        
        \,\leaders\hrule height 0.2pt\hfill\strut
        \vspace{0.2cm}
        
        с участием\;\leaders\hrule height 0.2pt\hfill\strut
        \vspace{0.1cm}
        
        рассмотрев материалы заявленного похода группы под руководством Чашниковой А. А. считает, что (ненужное зачеркнуть):

        \begin{enumerate}
            \item Маршрут \textbf{соответствует} (не соответствует) заявленной категории сложности.
            \item Туристский опыт руководителя группы \textbf{соответствует} (не соответствует) технической сложности маршрута.
            \item Туристский опыт участников группы \textbf{соответствует} (не соответствует) технической сложности маршрута.
            \item Заявочные материалы \textbf{отвечают} (не отвечают) установленным требованиям.
            \item Другие замечания:\;\leaders\hrule height 0.2pt\hfill\strut
            
            \,\leaders\hrule height 0.2pt\hfill\strut

            \,\leaders\hrule height 0.2pt\hfill\strut

            \,\leaders\hrule height 0.2pt\hfill\strut
        \end{enumerate}
        
        Группе назначается (\textbf{не назначается}) контрольная проверка на местности\;\leaders\hrule height 0.2pt\hfill\strut
        \vspace{0.2cm}
        
        \,\leaders\hrule height 0.2pt\hfill\strut
        \vspace{0.2cm}
        
        \,\leaders\hrule height 0.2pt\hfill\strut
        \vspace{0.2cm}
        
        \,\leaders\hrule height 0.2pt\hfill\strut
        
        \hbox to \textwidth{\hfil\raisebox{1pt}{\scriptsize(где, когда и по каким вопросам)}\hfil}

\newpage %---------------двенадцатая страница------------------------------

    \section[]{Результаты проверки группы на\\ местности}

        Группа в составе 7 человек прошла проверку 9\;июня\;2025 года; кросс-поход т \ к <<Вестра>>, класс <<А>>
        \vspace{0.2cm}

        % \settowidth{\tmp}{Группа в составе: руководитель}
        % \ttmp = \textwidth
        % \addtolength{\ttmp}{-\tmp}
        
        % Группа в составе: руководитель%
        % \rlap{\raisebox{-0.3cm}{\hbox to \ttmp{\hfil\scriptsize Фамилия И.О.\hfil}}}%
        % \;\leaders\hrule height 0.2pt\hfill\strut
        % \vspace{0.2cm}

        % \settowidth{\tmp}{участники}
        % \ttmp = \textwidth
        % \addtolength{\ttmp}{-\tmp}
        
        % участники%
        % \rlap{\raisebox{-0.3cm}{\hbox to \ttmp{\hfil\scriptsize Фамилия И.О.\hfil}}}%
        % \;\leaders\hrule height 0.2pt\hfill\strut
        % \vspace{0.2cm}
        
        % \,\leaders\hrule height 0.2pt\hfill\strut
        % \vspace{0.2cm}
        
        % прошла проверку\;%
        % <<\vbox{\hbox{\hphantom{999}}\hrule height 0.2pt}>>\;\hbox to 2.5cm{\leaders\hrule height 0.2pt\hfil}\;%
        % 20\,\vbox{\hbox{\hphantom{999}}\hrule height 0.2pt}\,г.%
        % \rlap{\raisebox{-0.3cm}{\hbox to 0.28\textwidth{\hfil\scriptsize место проведения\hfil}}}%
        % \hbox to 0.28\textwidth{\;\leaders\hrule height 0.2pt \hfill}
        % \vspace{0.05cm}
        
        по следующим вопросам:\;\leaders\hrule height 0.2pt\hfill\strut
        \vspace{0.2cm}
        
        \,\leaders\hrule height 0.2pt\hfill\strut
        \vspace{0.2cm}
        
        Результаты проверки:\;\leaders\hrule height 0.2pt\hfill\strut
        \vspace{0.2cm}
        
        \,\leaders\hrule height 0.2pt\hfill\strut
        \vspace{0.2cm}
        
        Проверяющий\;\hbox to 0.3\textwidth{\leaders\hrule height 0.2pt \hfill}%
        \rlap{\raisebox{-0.3cm}{\hbox to 0.42\textwidth{\hfil\scriptsize Фамилия И.О.\hfil}}}%
        \hbox to 0.42\textwidth{\;(\;\leaders\hrule height 0.2pt \hfill\;)}
        \vspace{0.1cm}

    \section[]{Заключение маршрутно-\\квалификационной комиссии}

        Группа под руководством \textbf{%
        Чашниковой Анастасии Александровны} имеет (не имеет) положительное
        заключение МКК о прохождении планируемого маршрута.
        \vspace{0.2cm}
        
        Особые указания:\;\leaders\hrule height 0.2pt\hfill\strut
        \vspace{0.2cm}
        
        \,\leaders\hrule height 0.2pt\hfill\strut
        \vspace{0.2cm}
        
        \,\leaders\hrule height 0.2pt\hfill\strut
        \vspace{0.2cm}
        
        \,\leaders\hrule height 0.2pt\hfill\strut
        \vspace{0.2cm}
        
        Срок сдачи отчета о походе до\;%
        <<\vbox{\hbox{\hphantom{999}}\hrule height 0.2pt}>>\;\hbox to 3cm{\leaders\hrule height 0.2pt\hfil}\;%
        20\,\vbox{\hbox{\hphantom{999}}\hrule height 0.2pt}\,г.%
        
        в объеме \textbf{письменный, полный}
        \vspace{0.2cm}
        
        Адреса и реквизиты для связи с региональным органом МЧС, ПСС или ПСО в районе проведения маршрута:%
        \;\leaders\hrule height 0.2pt\hfill\strut
        \vspace{0.2cm}

        \,\leaders\hrule height 0.2pt\hfill\strut
        \vspace{0.2cm}

\newpage %---------------тринадцатая страница------------------------------

    \section{Контрольные пункты и сроки}

        О прохождении маршрута группа должна сообщить:

        \begin{enumerate}[itemsep=1pt]
            \item \parbox[t]{0.58\textwidth}{%
            Штаб КЧС по ГБАО} \hfil \parbox[t]{0.325\textwidth}{\raggedleft(83522) 2-40-57, 2-53-47}
            \item Зеленцову \hfill +79057863310
\item Варгафтику \hfill +79032748883

        \end{enumerate}

        \begin{itemize}[label=из, noitemsep]
            \item Душанбе до 24.7.2025\;г.
            \item Душанбе до 18.8.2025\;г.
        \end{itemize}

        \parbox[t]{0.4\textwidth}{\raggedright Номер телефона координатора группы:} \hfil \parbox[t]{0.5\textwidth}{\raggedleft Гриша (до 9.08) +79032748883 \\ @byabik,\\\vspace{0.2cm} Катя (после 9.08) +79258523626 \\ @Kropocheva\_Ekaterina}
        \vspace{0.4cm}

        \hbox to \textwidth{%
        \textbf{Председатель МКК}\hfil\hbox to 0.2\textwidth{\leaders\hrule height 0.2pt \hfill}%
        \rlap{\raisebox{-0.3cm}{\hbox to 0.42\textwidth{\hfil\scriptsize Фамилия И.О.\hfil}}}%
        \hbox to 0.42\textwidth{\;(\;\leaders\hrule height 0.2pt \hfill\;)}}
        \vspace{0.4cm}

        \hbox to \textwidth{%
        \textbf{Члены МКК}\hfil\hbox to 0.2\textwidth{\leaders\hrule height 0.2pt \hfill}%
        \rlap{\raisebox{-0.3cm}{\hbox to 0.42\textwidth{\hfil\scriptsize Фамилия И.О.\hfil}}}%
        \hbox to 0.42\textwidth{\;(\;\leaders\hrule height 0.2pt \hfill\;)}}
        \vspace{0.2cm}

        \hbox to \textwidth{\hfil%
        \hbox to 0.2\textwidth{\leaders\hrule height 0.2pt \hfill}%
        \rlap{\raisebox{-0.3cm}{\hbox to 0.42\textwidth{\hfil\scriptsize Фамилия И.О.\hfil}}}%
        \hbox to 0.42\textwidth{\;(\;\leaders\hrule height 0.2pt \hfill\;)}}
        \vspace{0.2cm}

        \hbox to \textwidth{\hfil%
        \hbox to 0.2\textwidth{\leaders\hrule height 0.2pt \hfill}%
        \rlap{\raisebox{-0.3cm}{\hbox to 0.42\textwidth{\hfil\scriptsize Фамилия И.О.\hfil}}}%
        \hbox to 0.42\textwidth{\;(\;\leaders\hrule height 0.2pt \hfill\;)}}
        \vspace{0.2cm}

        {\small%
        \textbf{Судья по виду:} <<Сертификаты на знание антидопинговых правил и страховые полисы предъявлены,
        меддопуск имеется. Группа допущена к соревнованиям>>%
        \rlap{\raisebox{-0.3cm}{\hbox to 0.65\textwidth{\hfil\scriptsize Статус и наименование соревнований\hfil}}}%
        \hbox to 0.65\textwidth{\;\leaders\hrule height 0.2pt \hfill}}
        \vspace{0.5cm}

        Штамп МКК
        \vspace{0.2cm}

        \hfill%
        \rlap{\raisebox{-0.3cm}{\hbox to 0.3\textwidth{\hfil\scriptsize подпись\hfil}}}%
        \hbox to 0.3\textwidth{\leaders\hrule height 0.2pt \hfill}%
        \rlap{\raisebox{-0.3cm}{\hbox to 0.3\textwidth{\hfil\scriptsize Фамилия И.О.\hfil}}}%
        \hbox to 0.3\textwidth{\quad\leaders\hrule height 0.2pt \hfill}
        \vspace{0.5cm}

        \hfill<<\vbox{\hbox{\hphantom{999}}\hrule height 0.2pt}>>\;\hbox to 3cm{\leaders\hrule height 0.2pt\hfil}\;%
        20\,\vbox{\hbox{\hphantom{999}}\hrule height 0.2pt}\,г.%

\newpage %---------------пятнадцатая страница------------------------------

    \section[]{Регистрация в территориальном органе\\ МЧC}

        \textbf{Регистрационный номер}\;\leaders\hrule height 0.2pt\hfill\strut
        \vspace{0.4cm}

        \textbf{Дополнительные отметки}\;\leaders\hrule height 0.2pt\hfill\strut
        \vspace{0.2cm}

        \hbox to \textwidth{%
        \;\leaders\hrule height 0.2pt \hfill}
        \vspace{0.2cm}

        \hbox to \textwidth{%
        \;\leaders\hrule height 0.2pt \hfill}
        \vspace{1cm}

        \settowidth{\tmp}{Штамп ПСС (ПСО)}

        \hfill\hbox{\vbox{\hbox{Штамп ПСС (ПСО)}
        \rlap{\raisebox{0cm}{\hbox to \tmp{\hfil\footnotesize (при наличии)\hfil}}}}}
        \vspace{2cm}

    \section{Решение МКК о зачёте маршрута}

        Пройденный группой \textbf{горный} маршрут под руководством Чашниковой А. А. оценен
        \;\leaders\hrule height 0.2pt \hfill\;категорией сложности.
        \vspace{1cm}

        \hfill Справки выданы в количестве\;\hbox to 2cm{\leaders\hrule height 0.2pt \hfill}\;шт.
        \vspace{2cm}

        \hbox to \textwidth{%
        \textbf{Председатель МКК}\hfil\rlap{\raisebox{-0.3cm}{\hbox to 0.3\textwidth{\hfil\scriptsize подпись\hfil}}}%
        \hbox to 0.3\textwidth{\leaders\hrule height 0.2pt \hfill}%
        \rlap{\raisebox{-0.3cm}{\hbox to 0.3\textwidth{\hfil\scriptsize Фамилия И.О.\hfil}}}%
        \hbox to 0.3\textwidth{\;(\;\leaders\hrule height 0.2pt \hfill\;)}}
        \vspace{1cm}

        \hbox to 0.35\textwidth{\hfil Штамп МКК}
        \vspace{1cm}

        \hbox to \textwidth{\hfil<<\vbox{\hbox{\hphantom{999}}\hrule height 0.2pt}>>\;\hbox to 3cm{\leaders\hrule height 0.2pt\hfil}\;%
        20\,\vbox{\hbox{\hphantom{999}}\hrule height 0.2pt}\,г.}

\newpage %---------------шестнадцатая страница-----------------------------

    \strut

\newpage %---------------семнадцатая страница------------------------------

    {\centering%
    \textbf{Положение о МКК\\
    (Извлечение)}\par}
    \vspace{1cm}

    \textbf{5.7. При рассмотрении заявочных документов на походы МКК обязаны
    проверить:}

    \begin{itemize}[itemsep=2pt, leftmargin=1.1cm, label=$\triangleright$]
        \item разработку маршрута и график движения группы по основному и запасным вариантам,
        наличие картографических материалов;
        \item знание руководителем группы района похода условий передвижения и естественных препятствий в нем;
        \item соответствие туристского опыта руководителя и участников похода заявленному маршруту;
        \item правильность подбора группой снаряжения, продовольствия, медикаментов;
        \item намеченные группой меры по обеспечению безопасности при проведении похода;
        \item правильность выбора контрольных пунктов и сроков;
    \end{itemize}

    Заявочные документы регистрируются и хранятся в организации, при которой создана МКК, не менее трех лет.
    \vspace{0.5cm}

    \textbf{5.8. МКК имеют право:}

    \begin{itemize}[itemsep=2pt, leftmargin=1.1cm, label=$\triangleright$]
        \item вызвать участников группы и проверить знание ими Правил проведения туристских
        спортивных походов, вопросов техники и тактики похода;
        \item назначить группе контрольный выход, где проверяется умение пользоваться снаряжением,
        преодолевать естест\-венные препятствия и действовать в аварийных ситуациях.
    \end{itemize}

\newpage %---------------восемнадцатая страница---------------------------

\strut

\newpage %---------------девятнадцатая страница---------------------------

% Пустая

\newpage %---------------двадцатая страница-------------------------------

\strut

\newpage %---------------двадцать первая страница--------------------------

\strut

\end{document}